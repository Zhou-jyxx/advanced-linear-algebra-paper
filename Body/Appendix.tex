\section{附录}

\subsection{环境要求}
Python >= 3

Numpy == 1.18.1

Matplotlib == 3.1.3

\subsection{梯度下降法求解最小二乘问题}
\begin{lstlisting}
import numpy as np
import matplotlib.pyplot as plt
np.random.seed(1234)

len_x = 1000
A = np.random.rand(len_x, len_x)
x_star = np.random.rand(len_x, 1)
b = A*x_star
my_x = np.random.rand(len_x, 1)

def get_grad(x):
    return 2*A.T*(A*x-b)

def get_loss(x):
    return np.linalg.norm(A*x-b, ord=1)**2

delta = 1e10
alpha = 5e-1
iter = 0
iter_list = []
delta_list = []
# while delta > 1e-4:
# 手动设置迭代1000次
while iter<1e3:
    new_x = my_x - alpha*get_grad(my_x)
    delta = np.linalg.norm(new_x - x_star, ord=2) / np.linalg.norm(x_star, ord=2)
    iter_list.append(iter)
    delta_list.append(np.log(delta))
    print(f'{iter}, loss is {get_loss(my_x)}, rela_x is {delta}')
    my_x = new_x
    iter += 1

plt.plot(iter_list, delta_list)
plt.xlabel('epoch')
plt.ylabel('log ||newx-x||')
plt.savefig('gd-loss.png', dpi=400)
\end{lstlisting}

\subsection{临近梯度下降法求解Lasso问题}
\begin{lstlisting}
import numpy as np
import matplotlib.pyplot as plt

def generate_data(m=1000, n=100, density=0.4):
    "Generates data matrix X and observations Y."
    np.random.seed(1)
    x_star = np.random.random((m,1))
    idxs = np.random.choice(range(m), int((1-density)*m), replace=False)
    for idx in idxs:
        x_star[idx] = 0
    A = np.random.random((n, m))
    b = A.dot(x_star)
    return x_star, A, b


m, n, density = 1000, 100, 0.2
mu = 1
x_star, A, b = generate_data(m, n, density)
my_x = np.random.random((m, 1))


def f(A, x1, x2, b, mu):
    """
    对于临近梯度下降法,x1=x2,对于ADMM,将目标问题拆分为两部分
    """
    return 0.5 * np.linalg.norm(np.dot(A, x1) - b, ord=2) + mu * np.linalg.norm(x2, ord=1)


def get_gradient(A, x, b):
    return np.dot(A.transpose(), np.dot(A, x) - b)


def get_prox(learning_rate, x, mu):
    maxPart = np.abs(x) - learning_rate * mu
    
    for i in range(len(maxPart)):
        maxPart[i] = 0 if maxPart[i] <= 0 else maxPart[i]
    
    return np.sign(x) * maxPart


def step_proximal(A, x, b, mu, learning_rate):
    return get_prox(learning_rate, x - learning_rate * get_gradient(A, x, b), mu)


def step_admm(A, x1, x2, b, mu, beta, lamb, rho):
    size = np.shape(A)[1]
    newX1 = np.dot(np.linalg.inv(np.dot(A.transpose(), A) + beta * np.identity(size)), np.dot(A.transpose(), b) + beta * x2 - lamb)
    newX2 = get_prox(mu/beta, newX1+1/beta*lamb, mu)
    newLamb = lamb + rho * beta * (newX1 - newX2)

    return newX1, newX2, newLamb

def proximal_gradient_descent(A, x, b, mu, learning_rate):
    delta = 1e10
    epoch = 0
    iterationList = []
    deltaList = []

    while delta > 1e-04 and epoch < 300:
        newX = step_proximal(A, x, b, mu, learning_rate)
        delta = np.abs(f(A, newX, newX, b, mu))
        iterationList.append(epoch)
        deltaList.append(delta)
        x = newX
        epoch += 1
    
    curve, = plt.plot(iterationList, deltaList, '-')
    return curve
\end{lstlisting}

\subsection{交替方向乘子法求解Lasso问题}
\begin{lstlisting}
def admm(A, x, b, mu, beta, lamb, rho):
    x1 = x
    x2 = x
    lamb = lamb * np.ones((np.shape(A)[1], 1))
    delta = 1e10
    epoch = 0
    iterationList = []
    deltaList = []

    while delta > 1e-04 and epoch < 50:
        newX1, newX2, newLamb = step_admm(A, x1, x2, b, mu, beta, lamb, rho)
        delta = np.abs(f(A, newX1, newX2, b, mu))
        iterationList.append(epoch)
        deltaList.append(delta)
        x1 = newX1
        x2 = newX2
        lamb = newLamb
        epoch += 1

    curve, = plt.plot(iterationList, deltaList, '-')
    return curve
\end{lstlisting}