\section{定理等环境测试}这节用来展示定理,引理等常用论文环境。
 
% \subsection{编号环境与不编号环境}

\subsection{定理环境}

\begin{theorem}\label{thm2_1}

    设$A,B$是两个实数, 则$2AB\leq 2 A^2+B^2$.
    
\end{theorem}

\begin{definition}[函数性质]

    函数的梯度是
    
\end{definition}

\begin{proof}
    这里是证明。
\end{proof}

\subsection{引理环境}

\begin{lemma}[Nakayama引理]\label{lem2_2}
    这是一条引理测试。。。
\end{lemma}

\subsection{问题环境}

\begin{problem}[连续统假设]是否存在$\mathbb{R}$的子集S使得$card(\mathbb{N})<card(S)<card(\mathbb{R})$?
\end{problem}
\begin{solution}
    不存在。
\end{solution}

\subsection{举例环境}

\begin{example}
    这是一个举例。
\end{example}

% \subsubsection{无编号环境}

% \begin{theorem*}

%     设$A,B$是两个实数, 则$2AB\leq 2 A^2+B^2$.
    
% \end{theorem*}

% \begin{proof}
%     这里是证明。
% \end{proof}

% \begin{lemma*}[Nakayama引理]
%     这是一条引理测试。。。
% \end{lemma*}

% \begin{problem*}[连续统假设]是否存在$\mathbb{R}$的子集S使得$card(\mathbb{N})<card(S)<card(\mathbb{R})$?
% \end{problem*}
% \begin{solution}
%     不存在。
% \end{solution}