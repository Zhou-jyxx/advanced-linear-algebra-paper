\thispagestyle{fancy}
\newcommand{\TitleCHS}{线性代数进阶课程} %中文标题

\newcommand{\TitleENG}{Advanced Linear Algebra Course} %英文标题

\newcommand{\Author}{姓名} %作者名字

\newcommand{\StudentID}{学号} %学号

\newcommand{\Department}{计算机科学与技术学院} %学院

\newcommand{\Major}{计算机科学与技术} %专业

\newcommand{\Supervisor}{方发明} % 任课教师名字

\newcommand{\CompleteYear}{2022} % 完成年份
\newcommand{\CompleteMonth}{07} % 完成月份

\newcommand{\KeywordsCHS}{关键词1,关键词2,关键词3,关键词4,关键词5} %中文关键词

\renewcommand\abstractname{\sffamily\zihao{-3} 摘要}
\phantomsection
\begin{abstract}
	\addcontentsline{toc}{section}{摘要}
	\zihao{5}\rmfamily
	\vspace{\baselineskip}
    在计算机广为普及的今天,最优化方法已然成为工程技术人员研究大规模优化问题所必备的研究工具。凸优化,作为最优化的一个子领域,研究定义于凸集中的凸函数最小化的问题。由于近年来计算机算力的提高和最优化理论的充分发展,凸优化问题以其优秀的性质在机器学习领域中拥有广泛的应用。本文针对凸优化模型展开研究,首先罗列了一些基础的数学概念,并对于所涉及的定理给予了详细的证明。然后分别介绍了牛顿迭代法、拟牛顿方法、梯度下降法、随机梯度类方法、共轭梯度法、临近梯度下降法、增广拉格朗日法和交替方向乘子法这八种优化方法。在每个方法中,首先描述了该方法适用的凸优化模型,然后介绍了其对应的算法流程和具体的迭代格式,接着基于方法进行收敛性或优缺点分析,最后列举了如Lasso问题、最小二乘问题、矩阵分解问题等经典模型,以具体的应用实例来增加读者对于方法在实际机器学习问题中使用的理解。
	
	关键词: 凸优化,机器学习,牛顿法,梯度法,乘子法
\end{abstract}